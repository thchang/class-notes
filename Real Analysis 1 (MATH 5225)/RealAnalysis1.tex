% preamble
\documentclass[12pt]{article}
\usepackage{geometry, amsmath, amssymb, caption, subcaption, enumerate}
\geometry{
letterpaper,
total={6.5in,9in},
left=1in,
top=1in}
\parindent = 0pt
\parskip = 6pt

% document
\begin{document}
\section*{Real Analysis 1}
Tyler Chang\\
\today

\subsection*{Measure Theory}
 
How does one measure the ``size'' or ``mass'' of a set $S$ of elements from 
some space $X$?
If $X=\mathbb{R}^n$ and $S$ is defined by a geometric shape, it makes sense
to take the volume of $S$ as its measure.
But in a more general sense, this won't always work, especially when 
$X \neq \mathbb{R}^n$.

A {\it measurable} space is a pair $(X,M)$ of a set $X$ and a family $M$ of 
subsets of $X$.
A {\it measure} space is a triplet $(X,M,\mu)$ where 
$\mu : M \rightarrow [0,\infty]$ is a function that measures the size of
elements of $M$.
Note that in general (unless $M$ is the power set on $X$), $\mu$ cannot measure
every subset of $X$, it can only measure elements in the set $M$ of measurable 
subsets of $X$.

In order for $(X,M)$ to qualify as a measure space, $M$ must be a 
$\sigma$-algebra on $X$.
That is, $M$ must satisfy the following axioms:
\begin{itemize}
\item $\emptyset \in M$ and $X \in M$.
\item $M$ is closed under countable unions.
\item $M$ is closed under complements.
\end{itemize}
By De'Morgan, it follows that $M$ must also be closed under intersections.
Note that if $M$ has a countable basis, then $M$ is a topology, but not a
very interesting one since every set is clopen.

Given a measurable space $(X,M)$, in order for $(X,M,\mu)$ to define a 
measure, $\mu$ must satisfy the measure axioms:
\begin{itemize}
\item $\mu(\emptyset) = 0$
\item $\mu$ is countably additive. That is, for every countable collection
of pairwise disjoint sets $\{E_i\}_{i=1}^\infty$ with $E_i \in M$, 
$\mu(\cup_{i=1}^\infty E_i) = \sum_{i=1}^\infty \mu(E_i)$.
\end{itemize}
The measure axioms imply several other properties, including monotonicity: 
If $E \subset F$, then $\mu(E) \leq \mu(F)$.

A set $N \in M$ is a null set if $\mu(N) = 0$.
If $E \subset N$, then $E$ is a subnull set.
A measure space is complete if it contains all its subnull sets.
I.e., if $N$ is a subnull set, then $N \in M$.
If something is true everywhere except on a null set, then real analysis 
techniques cannot distinguish it from being true everywhere.
To cover these corner cases, we say that some things are true 
{\it almost everywhere}.
For example, $f = g$ almost everywhere if $f(x) = g(x)$ for all 
$x \in X \setminus N$ where $\mu(N)=0$.

If we relax the second measure axiom, allowing that $\mu$ need only be
subadditive (we must now also enforce that $\mu$ is monotone as well), then
we get an outer measure $\mu^*$.
Every measure corresponds to an outer measure on the entire space $X$.
Then the Caratheodory criterion states that a set $E$ is $\mu$-measurable
(that is, $E \in M$, for $(X,M,\mu)$ where $\mu = \mu^*|_M$) if and only if
$$
\mu^*(A) = \mu^*(E \cap A) + \mu^*(E^C \cap A)
\quad\text{ for all $A \subseteq X$.}
$$

In the special case where $X = \mathbb{R}^n$, $M$ is the $\sigma$-algebra
generated by boxes in $\mathbb{R}^n$ (i.e., $M$ is the coarsest $\sigma$-algebra
containing every open or closed box in $\mathbb{R}^n$, including single points),
and $\mu$ is defined as the unique measure that assigns the volume of a box
as its measure, we get the standard measure space on $\mathbb{R}^n$,
denoted $(\mathbb{R}^n, \mathcal{L}, m)$ where $m$ is the {\it Lebesgue measure}
and $\mathcal{L}$ are the {\it Lebesgue measurable sets}.

\subsection*{Measurable Functions and Integration}

Let $f : X \rightarrow \mathbb{R}^n$ be a function on a measure space 
$(X,M,\mu)$.
Then $f$ is measurable if $f^{-1}(U) \in M$ for all open $U$ in $\mathbb{R}^n$ 
with the standard topology.

A special type of measurable function is the {\it indicator function} 
$1_E : X \rightarrow \{0,1\}$ which returns $1_E(x) = 1$ if $x\in E$ and 
$1_E(x) = 0$ otherwise.
Another class of measurable functions are the {\it simple functions}, which are 
finite linear combinations of indicator functions:
$$
f_{simp}(x) = \sum_{i=1}^n c_i 1_{E_i}(x)
$$
where $E_i$ are all measurable.
We define the simple integral on simple functions by:
$$
Simp \int_X f_{simp}(x) d\mu = \sum_{i=1}^n c_i \mu(E_i).
$$
Note that the simple integral is both monotone (if $f(x) \leq g(x)$ for all
$x \in X$ then $\int_X f\text{ }d\mu \leq \int_X g\text{ }d\mu$) and linear
($\int_X (f+g)\text{ }d\mu = \int_X f\text{ }d\mu + \int_X g\text{ }d\mu$).

In general, we define the usigned integral for an unsigned measurable function 
$f$ as:
$$
\int_X f\text{ }d\mu = \sup_{g \leq f\text{, }g \in Simp(X)} 
Simp \int_X g\text{ }d\mu.
$$
then we define the signed integral for a general measurable function $f$:
$$
\int_X f\text{ }d\mu = \int_X f^+ d\mu - \int_X f^- d\mu
$$
where $f^+$ and $f^-$ denote the positive and negative components of $f$
respectively.
We define integration even more generally for complex-valued functions by 
decomposing $f$ into real and complex parts similarly as above.

If $\int_X |f| d\mu < \infty$ then we say that $f$ is absolutely integrable
(or just integrable) and place it in the $\mathcal{L}^1$ class of functions.
This induces a norm on $f$, specifically, 
$\| f \|_{L^1} = \int_X |f|\text{ }d\mu$.

\subsection*{Integration Theory}

Recall from elementary multivariable calculus, that we can exchange the order
of integration for ``most'' functions, but there are some strange functions
where one cannot.
Let $f : X \times Y \rightarrow \mathbb{C}$ where $(X, M_X, \mu_X)$ and 
$(Y, M_Y, \mu_Y)$ are measure spaces and $X \times Y$ gets the product measure 
space (i.e., the coarsest measure space where all $E \times F$ are measurable 
with $E\in M_X$ and $F \in M_Y$, and the measure is given by 
$\mu_X(E)\mu_Y(F)$).

Consider the integral:
$\int_{X\times Y} f \text{ }dx\text{ }dy.$
We want to say that
$$
\int_{X\times Y} f \text{ }dx\text{ }dy
=\int_X \int_Y f\text{ }dy\text{ }dx
=\int_Y \int_X f\text{ }dx\text{ }dy 
$$
but first we must show when it is true that 
$$
\int_X \int_Y f\text{ }dy\text{ }dx
=\int_Y \int_X f\text{ }dx\text{ }dy 
$$

\underline{Key question}: When can we exchange the order of a limit and an 
integral?

Answering this question tells us a lot, since integrals can be decomposed into
limits of sums of simple functions.

\textbf{Monotone Convergence Theorem} (MCT):\\
Let $\{f_n\}_{n=1}^\infty$ be measurable and unsigned non-decreasing functions
(i.e., $f_1 \leq f_2 \leq f_3 \leq \ldots$).
Then 
$$
\lim_{n\rightarrow\infty} \int_X f_n d\mu 
= \int_X \lim_{n\rightarrow\infty} f_n d\mu.
$$
Note: The above has an analogue for nested sequences of sets 
$$ E_n \uparrow \quad \Rightarrow 
\lim_{n\rightarrow\infty}\mu(E_n) = \mu(\lim_{n\rightarrow\infty} E_n),$$ 
and similarly for $E_n \downarrow$.

\textbf{Dominated Convergence Theorem} (DCT):\\
Let $(X, M, \mu)$ be a measure space.
Let $\{f_n\}_{n=1}^\infty : X \rightarrow \mathbb{R}$ be measurable
such that $f_n \rightarrow f$ pointwise almost everywhere and
suppose there exists an unsigned function $g \in \mathcal{L}^1$ that dominates 
$f_n$ almost everywhere ($|f_n(x)| \leq g(x)$ except on a null set).
Then 
$$
\lim_{n\rightarrow\infty} \int f_n \text{ }d\mu
= \int \lim_{n\rightarrow\infty} f_n \text{ }d\mu 
= \int f\text{ } d\mu.
$$

\textbf{Tonelli's Theorem and Fatou's Lemma}:\\
Let $\{f_i\}_{i=1}^\infty$ unsigned be measurable.
Then
$$
\int_X \left(\sum_{n=1}^\infty f_n\right)\text{ }d\mu 
= \sum_{n=1}^\infty \int_X f_n\text{ }d\mu. \qquad\text{(Tonelli)}
$$
and
$$
\int \lim \inf_{n\rightarrow\infty} f_n\text{ }d\mu 
\leq \lim\inf_{n\rightarrow\infty} \int f_n\text{ }d\mu.\qquad\text{(Fatou)}
$$

Using the above Theorems, we arrive at the crux of Real Analysis:

First, we arrive at the conclusion that integration by parts is well-defined
for all unsigned measurable functions. This is a direct consequence of
Tonelli's Theorem.
Then, if $f$ is signed, we get:

\textbf{Fubini's Theorem}:\\
If $f : M_X \times M_Y \rightarrow \mathbb{C}$ is absolutely integrable 
(in $\mathcal{L}^1$) then the integral is well-defined and
$$
\int_{X\times Y} f \text{ }dx\text{ }dy
=\int_X \int_Y f\text{ }dy\text{ }dx
=\int_Y \int_X f\text{ }dx\text{ }dy.
$$

In conclusion, note that integration by parts works if $f$ is {\it either}
unsigned {\it or} absolutely integrable.
In other words, integration by parts breaks down when $f$ has both positive
and negative parts and its integral is infinite.

\subsection*{Littlewood's Three Principles}

These require some interpretation, but give some intuition about the
properties of measurable sets and functions.
To interpret the word ``nearly,'' think 
``can be approximated to arbitrary precision by'' or 
``except on a null set'' as appropriate:
\begin{enumerate}
\item Every measurable set is nearly a finite union of intervals.
\item Every absolutely integrable function is nearly continuous.
\item Every pointwise convergent sequence of functions is nearly uniformly
convergent.
\end{enumerate}
It should be noted that these principles apply primarily to the special case of 
$\mathbb{R}^n$ with the Lebesgue measure.

\subsection*{Signed Measures}

On a final note, though we have only discussed unsigned measures:
$\mu : M \rightarrow [0,+\infty]$, we can also allow for signed measures
$\sigma : M \rightarrow \mathbb{R}$.
However, every signed measure $\sigma$ can be decomposed into the difference
between two unsigned measures $\sigma^+$ and $\sigma^-$.
That is,
$\sigma = \sigma^+ - \sigma^-$.

\end{document}
