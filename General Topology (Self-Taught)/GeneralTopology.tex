% preamble
\documentclass[12pt]{article}
\usepackage{geometry, amsmath, amssymb, caption, subcaption, enumerate, upgreek}
\geometry{
letterpaper,
total={6.5in,9in},
left=1in,
top=1in}
\parindent = 0pt
\parskip = 6pt

% document
\begin{document}
\section*{General Topology}
Tyler Chang\\
\today

\subsection*{Basic Definitions}

A topology is a pair $(X, \mathcal{T})$ where $X$ is a set and $\mathcal{T}$ is a set of
subsets of $X$ such that $\mathcal{T}$ satisfies three properties:
\begin{itemize}
\item $X \in \mathcal{T}$ and $\emptyset \in \mathcal{T}$.
\item $\mathcal{T}$ is closed under arbitrary unions. I.e., if some 
(possibly uncountable) family of sets $\{U_\alpha\}\in \mathcal{T}$, then 
$\cup U_\alpha \in \mathcal{T}$.
\item $\mathcal{T}$ is closed under finite intersections.
I.e., if $U, V \in \mathcal{T}$, then $U \cap V \in \mathcal{T}$.
\end{itemize}

The elements of $\mathcal{T}$ are called {\it open sets} under the topology 
$(X, \mathcal{T})$, and the complements of open sets are called 
{\it closed sets} under $(X, \mathcal{T})$.
A set is {\it clopen} if it is both open and closed (note, $\emptyset$ and $X$
are always clopen by definition).

An open set containing some $x \in X$ is called an open neighborhood of $x$.
The largest open set contained in a set $E$ is called the {\it interior} of $E$
and is denoted $E^\circ$.
The smallest closed set containing $E$ is called the {\it closure} of $E$ and
is denoted $\bar{E}$.
The {\it boundary} of a set $E$, denoted $\partial E$, is the set difference 
between the closure of $E$ and the interior of $E$.
A point is on the boundary of $E$ if and only if every open neighborhood of $x$
contains points both in $E$ and in $E^c$.
A set $E$ is dense in $X$ if $\bar{E} = X$, and nowhere dense if
$\left(\bar{E}\right)^\circ = \emptyset$.

A sequence of points $\{x_n\}$ in $X$ is said to {\it converge} to a point 
$x \in X$ if for every open neighborhood $U$ of $x$, there exists some value 
$N$ such that $x_n \in U$ for all $n > N$.
Then $x$ is said to be a {\it limit point} of $\{x_n\}$.
In general, the limit point of a sequence need not be unique.
However, in a Hausdorff space (described later) every limit is unique.
Using this terminology, the following are equivalent definitions of closedness:
\begin{itemize}
\item A set $C$ is closed (by definition) if it is the complement of an open
set.
\item A set $C$ is closed if and only if every convergent sequence of points 
in $C$ converges to a point in $C$ (i.e., $C$ contains all its limit points).
\item A set $C$ is closed if and only if it contains all its boundary points.
\end{itemize}

A topology $\mathcal{T}$ is said to be generated by a basis $\mathcal{B}$ if
all the elements of $\mathcal{T}$ can be written as unions of elements
in $\mathcal{B}$.
The standard topology on $\mathbb{R}^d$ is the topology generated by open
balls $B(x,r)$ where $B(x,r) = \{y \text{ : }  d(y,x) < r\}$ where 
$d(x,y) = \|x - y\|_2$.
In general, any topology generated by open balls with respect to some
metric $d$ is called a metric space.
A metric space $X$ is said to be {\it complete} if every Cauchy sequence:
($\{x_n\}$ such that $d(x_n, x_m) \rightarrow 0$ as $n,m \rightarrow \infty$)
converges to a point in $X$.

A set $K$ is compact if every open cover of $K$ (i.e., set of
open sets whose union contains $K$) has a finite subcover.
The {\it Heine-Borel Theorem} states that
in a finite-dimensional metric space, this is equivalent to stating that $K$ is
closed and bounded.
In an infinite-dimensional metric space, we need the further condition that $K$
is closed and bounded, and $K$ lies within some the $\varepsilon$-neighborhood
of some finite-dimensional metric space where $\varepsilon > 0$.
An entire space $X$ is compact if as a set, $X$ is compact under its own 
topology.

\subsection*{Continuity and Homeomorphisms}

A function $f : X \rightarrow Y$ is said to be {\it continuous} with respect to
$(X, \mathcal{T}_X)$ and $(Y,\mathcal{T}_Y)$ if $f^{-1}(U) \in \mathcal{T}_X$ 
for all in $U \in \mathcal{T}_Y$.

Note, that if we were to rename the elements of $X$ while maintaining the
same topology on the renamed set, then the change would be
superficial and, from a topological perspective, we wouldn't be able to
distinguish between the spaces.
For example,
$$
X = \{a, b, c\}, \qquad 
\mathcal{T}_X = \{ \emptyset, \{a\}, \{a,b\}, \{a,b,c\}\}
$$
is indistiguishable from
$$
Y = \{1, 2, 3\}, \qquad 
\mathcal{T}_Y = \{ \emptyset, \{1\}, \{1,2\}, \{1,2,3\}\}.
$$

We capture this notion of sameness by saying $X$ and $Y$ are {\it homeomorphic}.
If a continuous bijective function $h$ exists between $X$ and $Y$, then clearly
they are homeomorphic, and we call $h$ a {\it homeomorphism} between $X$ and 
$Y$.
Properties of a topology that are preserved under homeomorphisms are called
{\it topological invariants}.
For example, the existance of clopen sets other than $X$ and $\emptyset$ is a
topological invariant.

One important topological invariant is the {\it Hausdorff} property, which
states that for every pair of points $x_1,x_2 \in X$, there exists a
neighborhood $U_1$ of $x_1$ and $U_2$ of $x_2$ such that 
$U_1 \cap U_2 = \emptyset$.
Another important topological invariant is {\it separability}, defined as
the existence of a countable dense subset of $X$.
For example, $\mathbb{R}$ is separable under the standard topology since
the rationals are countable and desne in $\mathbb{R}$.

\subsection*{Manifolds}

A $n$-dimensional {\it manifold} is a topological space that ``looks like''
$n$-dimensional Euclidean space when we zoom in close enough.
To put this formally, every point $x$ in a manifold has an open neighborhood 
$U_x$ that is homeomorphic to $\mathbb{E}^n$ (that is, if we restrict the 
topology on $X$ to its intersection with $U_x$).
For example, the surface of the Earth is a $2$-dimensional manifold because
to a person standing on its surface, it looks like a flat plane.

\end{document}
