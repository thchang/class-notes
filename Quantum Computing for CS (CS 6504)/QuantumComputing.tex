% preamble
\documentclass[12pt]{article}
\usepackage{geometry, amsmath, amssymb, caption, subcaption, enumerate}
\geometry{
letterpaper,
total={6.5in,9in},
left=1in,
top=1in}
\parindent = 0pt
\parskip = 6pt

% document
\begin{document}
\section*{Quantum Computing}
Tyler Chang\\
\today

\subsection*{Fundamentals}

\begin{itemize}
	\item A {\it qubit} is a quantum bit, i.e., a unit with quantum
		properties and
		two logical states, usually denoted by upward and downward
		spin.
	\item It is often not possible to know the state of a qubit
		with certainty until it is sampled. When a qubit
		has non-zero probability of being in more than one state,
		it is said to be in a {\it superposition} of states.
	\item When a qubit is not in a superposition of states, i.e., it
		is known to be in a particular state with certainty, then
		it is said to be in a {\it pure} state.
	\item For a system of $n$ qubits, there are $2^n$ pure states.
		If 0 denotes a downward spin and 1 denotes an upward spin,
		then the pure states are denoted by
		$|0,0,\ldots,0,0\rangle$, $|0,0,\ldots,0,1\rangle$, $\ldots$,
		$|1,1,\ldots,1,1\rangle$.
	\item To observe the state of a qubit, a sample must be taken.
		Observing the state of a quantum system causes
		{\it collapse}, i.e., all the probabilities in the system
		are destroyed.
	\item Two quantum particles are said to be {\it entangled} when
		their probabilities are dependent. I.e., this can be thought
		of as an interaction between qubits.
	\item Qubits are capable of jumping across an energy barrier to
		a lower energy state through a process called 
		{\it quantum tunneling}. The probability of tunneling across
		a barrier decays with the width of the barrier.
	\item A {\it quantum computer} is a system of entangled qubits.
		Generally, to maintain the entangled states, the qubits
		must be stored at nearly absolute zero temperature and
		are extremely sensitive to vibrations.
\end{itemize}

\subsection*{Quantum Annealing}

A quantum annealer is a special purpose quantum computer thta relies on
quantum tunneling to minimize the energy of a quantum system.
The energy in a system of $n$ qubits $\sigma=[\sigma_1,\ldots,\sigma_n]^T$
can be expressed using the {\it Ising-Hamiltonian} model
$$
H(\sigma) = \sum_{i=1}^n\sum_{j=i+1}^n J_{i,j}\sigma_i^2\sigma_j^2
+ \sum_{i=1}^n h_i \sigma_i^2
$$
where $\sigma^2 \in \{-1, 1\}$ represents the spin of $\sigma_i$
($1$ indicates an upward spin, $-1$ indicates a downward spin),
$h_i$ indicates the energy associated with $\sigma_i$, and
$J_{i,j}$ indicates the interaction energy (via entanglement)
betweein $\sigma_i$ and $\sigma_j$.

The Ising model is {\it isomorphic} to the
{\it quadratic unconstrained binary optimization} (QUBO) model
$$
C(x) = \sum_{i=1}^n\sum_{j=i+1}^n b_{i,j} x_i x_j +
\sum_{i=1}^n a_i x_i
$$
via the transformation $\sigma_i^2 = 2x_i^2 - 1$,
$h_i = \frac{a_i}{2} + \frac{\sum_{j=1}^n b_{i,j}}{4}$,
$J_{i,j} = \frac{b_{i,j}}{4}$.
The QUBO model can also be written as $C(x) = xAx$ where $A$ is a symmetric
$n \times n$ matrix, with $A_{i,i} = a_i$ and $A_{i,j} = A_{j,i} = b_{i,j}/2$.

In order to encode a QUBO or Hamiltonian, it suffices to construct a truth
table for all input and output bits, where the valid states for the circuit
should result in minimum energy (or {\it ground}) states in the
Hamiltonian's energy landscape.
Meanwhile, invalid states for the circuit should result in {\it excited},
or higher energy states.

For example, consider the following list QUBO of constraints
for encoding an AND gate ($x_1 \wedge x_2 = x_3$).
\begin{align*}
& a_3 > 0\\
& a_2 = 0\\
& a_2 + a_3 + b_{2,3} > 0\\
& a_1 = 0\\
& a_1 + a_3 + b_{1,3} > 0\\
& a_1 + a_2 + b_{1,2} > 0\\
& a_1 + a_2 + a_3 + b_{1,2} + b_{1,3} + b_{2,3} = 0.
\end{align*}
To satisfy these constraints, we construct the following QUBO cost
function
$$
C(x) = 3x_3 + x_1 x_2 - 2 x_1 x_3 - 2 x_2 x_3.
$$
I.e., $a_1 = 0$, $a_2 = 0$, $a_3 = 3$, $b_{1,2} = 1$, 
$b_{1,3} = -2$, $b_{2,3} = -2$.
This results in the following table of energies.
\begin{center}
\begin{tabular}{c|c|c|c}
	$x_1$ & $x_2$ & $x_3$ & $C(x)$ \\
	\hline 
	0 & 0 & 0 & 0\\
	\hline 
	0 & 0 & 1 & 3\\
	\hline 
	0 & 1 & 0 & 0\\
	\hline 
	0 & 1 & 1 & 1\\
	\hline 
	1 & 0 & 0 & 0\\
	\hline 
	1 & 0 & 1 & 1\\
	\hline 
	1 & 1 & 0 & 1\\
	\hline 
	1 & 1 & 1 & 0\\
\end{tabular}\end{center}

Sometimes it is not possible to satisfy all the constraints to encode
a function.
In these cases, an {\it ancillary bit} (or ancilla bit) is introduced,
whose value is assigned specifically to make the problem solvable.
For example, the XOR gate ($x_1 \oplus x_2 = x_3$) requires one ancilla bit
($x_4$). This lends the following cost function:
$$
C(x) = x_1 + x_2 + x_3 + 4x_4 
	+ 2x_1x_2 - 2x_1x_3 - 4x_1x_4 
	- 2x_2x_3 - 4x_2x_4
	+ 4x_3x_4.
$$
which produces the valid states
$(x_1 = 0, x_2 = 0, x_3 = 0, x_4 = 0)$, 
$(x_1 = 0, x_2 = 1, x_3 = 1, x_4 = 0)$, 
$(x_1 = 1, x_2 = 0, x_3 = 1, x_4 = 0)$, and
$(x_1 = 1, x_2 = 1, x_3 = 0, x_4 = 1)$.
Note, this is equivalent to a half-adder.

The above Hamiltonians/QUBOs represent {\it logical models}, which assume
a fully connected {\it graphy topology}.
However, in real world annealing systems, the qubits are typically
connected according to some sparse topology, where only pairs of
cross entangled qubits can share an interaction term.
To get a {\it physical model} for a specific system from the logical model,
one must create {\it chains} of connected qubits (by encoding an equality
constraint between them in the Hamiltonian) until a sufficiently dense
graph topology is synthesized.
This physical model can be annealed on the quantum annealer hardware, which
tunnels toward the ground energy state during the {\it sampling time}.
In real world systems, the sampling time $s$ needed to converge
is given by $s = \mathcal{O}(\Delta^{-2})$,
where $\Delta$ is the minimum energy gap between stability points
(i.e., pure spin states).
Therefore, it is typical to use relatively short annealing times and
many independent runs to generate a distribution of results, with the
highest probability solution(s) being returned in a post-processing step.

It should be noted that in this formulation, there is no distinction between
running a circuit forward and backward (i.e., inverting).
For example, for the AND gate, one could ``pin'' the values $x_1$ and $x_2$
to some predetermined values on input in order to solve for $x_3$.
Alternatively, one could pin $x_3$, and find all the values of $x_1$ and $x_2$
that would satisfy $x_1 \wedge x_2 = x_3$.
Most notably, this implies that multiplication and factoring are equivalent
operations on a quantum annealer.

Also note that Hamiltonians/QUBOs are additive, in that for two
Hamiltonians/QUBOs $H_1$ and $H_2$ with solution sets $X_1$ and $X_2$,
$H_3 = H_1 + H_2$ will have the solution set $X_3 = X_1 \cap X_2$.
This allows for simple Hamiltonians to be combined to engineer complex
circuits.

\subsection*{Quantum Gate Model}

Typically, when using the term ``quantum computer,'' people refer to
the {\it quantum gate model}.
This model is more similar to the classical computing model, and can
solve a different class of problems from the quantum annealer.
A {\it general purpose quantum computer} is capable of applying quantum
logic gates in sequence to a systems of qubits, where these qubits
need not be confined to pure states, and are allowed to exist
in a superposition of states.

A system of $n$ qubits is generally expressed as a complex vector with
$2^n$ entries, using {\it bra-ket} notation:
\begin{itemize}
\item $| a \rangle$ denotes a column vector;
\item $\langle a |$ denotes the row vector which is the Hermitian
conjugate of $| a \rangle$;
\item $\langle a | b \rangle$ denotes the inner product of $|a\rangle$
and $|b\rangle$; and
\item $|a\rangle \langle b|$ denotes the outer product of $|a\rangle$
and $|b\rangle$.
\end{itemize}

The basis vectors for this complex vector space are the $2^n$ pure states
described previously.
The square of the modulus of each term $|a_i\rangle$ corresponds
to the probability that $|a\rangle$ is in the $i$th pure state.
For example, if $|a\rangle = [a_1, a_2, \ldots, a_{2^n}]^T$, then we
can infer that the probability that 
$\mathbb{P}\left(|a\rangle = |0,0,\ldots,0,0\rangle\right) = |a_1|^2$,
$\mathbb{P}\left(|a\rangle = |0,0,\ldots,0,1\rangle\right) = |a_2|^2$,
and so on.
It follows that every valid state vector must have norm one, since
the probabilities must sum to one.
The unit sphere in $\mathbb{C}^m$ upon which all quantum states reside
is called the {\it Bloch sphere}.
If we agree to renormalize after addition and scalar multiplication,
then the Bloch sphere can be considered as a vector space.

Since quantum operations must preserve information and probabilities,
every quantum gate can be represented as a unitary matrix applied to the
state vector.
Since many classical gates are not reversible, it is typical to introduce
ancillary qubits in a quantum circuit to preserve information that would
be lost through a classical circuit.
Since a sequence of unitary transformations is itself a
unitary transformation, every quantum program can be represented as a
single unitary transformation.
Some common quantum gates are:
\begin{itemize}
\item Pauli spin matrices (X, Y, and Z);
\item The Fredkin (CSWAP) matrix;
\item The Hadamard (H) matrix;
\item The Toffoli (CCNOT) and CX (CNOT) matrix.
\end{itemize}
A {\it universal set} of quantum gates, is a set of unitary matrices
that can be applied in some sequence to approximate any unitary matrix
(i.e., any quantum circuit) to arbitrary precision.

After a quantum circuit has been run, a pure state solution is observed
with probabilties determined by its posterior state vector.
Therefore, a quantum circuit generally must be run many times to build
a solution distribution.
Since the quantum system and all its probabilities collapse after each
observation, this involves a complete re-run of the quantum circuit.
By starting inputs in superpositions of states, this allows for inherent
parallelism.

A {\it fault-tolerant} quantum computer is one that is capable of
returning correct solutions with high certainty.
Theoretically, a fault-tolerant quantum computer is capable of performing
any operation that a classical computer can.
However, it is yet unclear whether {\it quantum polynomial time} (QP) is
a superset of classical polynomial time (P).
Two famous quantum algorithms that certainly improve over classical algortihms
are
\begin{itemize}
\item Grover's algorithm: performs a search of $n$ unordered items in a list
in $\mathcal{O}(\sqrt{n})$ time;
\item Shor's algorithm: performs integer factorization using a quantum
Fourier transform.
\end{itemize}

\end{document}
