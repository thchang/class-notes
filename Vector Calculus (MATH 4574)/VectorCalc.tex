% preamble
\documentclass[12pt]{article}
\usepackage{geometry, amsmath, amssymb, caption, subcaption, enumerate, upgreek}
\geometry{
letterpaper,
total={6.5in,9in},
left=1in,
top=1in}
\parindent = 0pt
\parskip = 6pt

% document
\begin{document}
\section*{Vector Calculus}
Tyler Chang\\
\today

\subsection*{Definitions}

\begin{itemize}
\item {\it Scalar field}: a function $f : V \rightarrow F$ where $V$ is a vector
space (usually $\mathbb{R}^2$ or $\mathbb{R}^3$) and $F$ is a field 
(usually $\mathbb{R}$).
\begin{itemize}
\item For $V = \mathbb{R}^3$, this is often denoted: $f(x,y,z)$.
\end{itemize}
\item {\it Vector field}: a function $F : V \rightarrow V$ where $V$ is a vector
space (usually $\mathbb{R}^2$ or $\mathbb{R}^3$).
\begin{itemize}
\item For $V = \mathbb{R}^3$, this is often denoted: 
$F(x,y,z) = P(x,y,z)\hat{i} + Q(x,y,z)\hat{j} + R(x,y,z)\hat{k}$,
where $\hat{i}$, $\hat{j}$, and $\hat{k}$ denote the unit basis vectors in the
$x$, $y$, and $z$ directions.
\end{itemize}
\item A curve $C$ in $\mathbb{R}^2$ or a surface $S$ in $\mathbb{R}^3$ is
{\it parameterized} by a function $\gamma(t)$ or $\gamma(u,v)$ if $\gamma$
traces $C$ or $S$ while $t$ or $(u,v)$ are in the domain of $\gamma$.
\begin{itemize}
\item A curve $C$ parameterized by $\gamma(t)$, $t \in [a,b]$ is {\it closed}
if $\gamma(a) = \gamma(b)$.
\item A curve $C$ is {\it simple} if it does not cross itself anywhere.
\item A set $D\subset\mathbb{R}^n$ is {\it connected} if every pair of points
in $D$ can be connected by a curve in $D$, and {\it simply connected} if $D$
is connected and every simple closed curve in $D$ encloses only points in $D$.
\end{itemize}
\item A curve $C$ is {\it positively oriented} if it is traced in the 
counterclockwise direction by $\gamma(t)$.
\item A surface $S$ with exactly two faces is {\it positively/outward/upward}
oriented if all of its normal vectors point outward from the surface.
\item For a scalar field $f$ on $\mathbb{R}^3$, the {\it gradient vector field}
is denoted by: $\nabla f(x,y,z) = \langle f_x, f_y, f_z \rangle$, where 
$f_x$, $f_y$, and $f_z$ denote the partial derivatives of $f$ w.r.t. $x$, $y$,
and $z$ respectively.
\item A vector field $F$ is said to be {\it conservative} if $F = \nabla f$ for
some scalar field $f$ (called the {\it potential function}).
\begin{itemize}
\item In a conservative vector field, the work done moving a particle around
a closed curve is 0. I.e., for a force $F$ and a closed curve $C$ parameterized
by $\gamma$, the work done is given by $\int_C F \cdot d\gamma = 0$.
\item In a conservative vector field, the work done in moving a particle from
point $a$ to $b$ is independent of the path $\gamma(t)$.
\end{itemize}
\item For a vector field 
$F(x,y,z) = P(x,y,z)\hat{i} + Q(x,y,z)\hat{j} + R(x,y,z)\hat{k}$, 
the {\it divergence} of $F$ is given by 
$$
div(F) := \nabla \cdot F 
= \frac{\partial P}{\partial x}+\frac{\partial Q}{\partial y}
+\frac{\partial R}{\partial z}.
$$
\begin{itemize}
\item Physical intuition: Let $F$ represent fluid flow in $\mathbb{R}^3$ and
let $p$ be a point in $\mathbb{R}^3$.
Then if $div(F(p)) > 0$, the net flow of the fluid (flux) is outward
at $p$; and if $div(F(p)) < 0$, the net flow of fluid is inward at $p$.
\end{itemize}
\item For a scalar field $f$, the {\it Laplacian} of $f$ is given by
$$
\bigtriangleup f := div( \nabla f)
= \frac{\partial^2 f}{\partial x^2}+\frac{\partial^2 f}{\partial y^2}
+\frac{\partial^2 f}{\partial z^2}.
$$
\item For a vector field
$F(x,y,z) = P(x,y,z)\hat{i} + Q(x,y,z)\hat{j} + R(x,y,z)\hat{k}$, 
the {\it curl} of $F$ is given by
$$
curl(F) := \nabla \times F
= \langle \frac{\partial R}{\partial y} - \frac{\partial Q}{\partial z}
\text{, } \frac{\partial P}{\partial z} - \frac{\partial R}{\partial x}
\text{, } \frac{\partial Q}{\partial x} - \frac{\partial P}{\partial y}
\rangle.
$$
\end{itemize}

\subsection*{Surface and Line Integrals}

In general, line and surface integrals are computed using the measure $\gamma$
defined by $d\gamma = |\gamma'|dt$, where $\gamma$ parameterizes the
line/surface.
\begin{itemize}
\item The {\it line integral} or {\it curve integral} of a scalar field
$f : \mathbb{R}^2 \rightarrow \mathbb{R}$ over a curve $C$ in $\mathbb{R}^2$
parameterized by $\gamma(t)$, $t \in [a,b]$ is given by:
$$
\int_C f d\gamma := \int_a^b f(\gamma(t))\|\gamma'(t)\|dt.
$$
\begin{itemize}
\item Note: for a curve $C$, the negative orientation of $C$ is denoted $-C$.
Let $\gamma(t)$ parameterize $C$ for $t\in[a,b]$.
Then $\phi(t) = \gamma(a+b-t)$ traces $-C$ for $t\in[a,b]$, and
$\phi$ $\int_{-C} f(x,y) d\phi = \int_C f(x,y) d\gamma$.
\item Physical intuition: if $\rho(x,y)$ gives the density of a wire at
all points $(x,y)$ defined over some curve $C \subset \mathbb{R}^2$, then
$m = \int_C \rho d\gamma$ gives the {\it mass} of the wire, and the 
{\it center of mass} is given by $(\bar{x}, \bar{y})$ where
$\bar{x} = \frac{1}{m}\int_C x\rho(x,y) d\gamma$ and 
$\bar{y} = \frac{1}{m}\int_C y\rho(x,y) d\gamma$.
\item Note that $M_y := x\rho(x,y)$ is called the {\it moment} about the
y-axis, and $M_x := y\rho(x,y)$ is called the moment about the x-axis.
\item All the above still hold for a curve $C\in\mathbb{R}^3$ and a scalar
field $f : \mathbb{R}^3 \rightarrow \mathbb{R}$.
\end{itemize}
\item The line integral of a vector field 
$F(x,y,z) = P(x,y,z)\hat{i} + Q(x,y,z)\hat{j} + R(x,y,z)\hat{k}$ over a curve
$C$ in $\mathbb{R}^3$ parameterized by $\gamma(t)$, for $(t) \in [a,b]$, is
given by
$$
\int_C F \cdot d\gamma := \int_a^b F(\gamma(t)) \cdot \gamma'(t) dt.
$$
\begin{itemize}
\item Note: Let $\gamma(t)$ parameterize $C$ for $t\in[a,b]$.
Then $\int_{-C} F \cdot d\gamma = -\int_C F \cdot d\gamma$.
\item Physical intuition: if $F(x,y,z)$ represents a force exerted on a
particle at position $(x,y,z)\in\mathbb{R}^3$ and $\gamma(t)$ parameterizes
the path travelled by that particle through space, then the work done in
moving the particle from $\gamma(a)$ to $\gamma(b)$ is given by 
$\int_C F\cdot d\gamma$.
\end{itemize}
\item The {\it surface integral} of a scalar field 
$f : \mathbb{R}^3 \rightarrow \mathbb{R}$ over a surface $S$ parameterized
by $\gamma(u,v)$, $(u,v) \in D$ is given by
$$
\int \int_S f(x,y,z) d\gamma 
:= \int \int_D f\left(\gamma(u,v)\right)\|\gamma_u \times \gamma_v\| dA,
$$
where $\gamma_u$ and $\gamma_v$ denote the partial derivatives of $\gamma$
w.r.t. $u$ and $v$ respectively, and $dA$ denotes the standard double integral
over the region $D \subseteq \mathbb{R}^2$.
\begin{itemize}
\item Note: $\int \int_D \|\gamma_u \times \gamma_v\| dA$ gives the surface
area of $S$.
\item Physical intuition: If $\rho(x,y,z)$ is a density function defined over
a surface $S$, then the mass of $S$ is given by 
$m = \int\int_S \rho(x,y,z) d\gamma$, and the center of mass is given by
$(\bar{x},\bar{y},\bar{z})$ where $\bar{x}$, $\bar{y}$, and $\bar{z}$ are
defined similarly as in the case of the wire.
\end{itemize}
\item The {\it surface integral} of a vector field 
$F(x,y,z) = P(x,y,z)\hat{i} + Q(x,y,z)\hat{j} + R(x,y,z)\hat{k}$ over a
surface $S$ parameterized by $\gamma(u,v)$, $(u,v) \in D$ is given by
$$
\int \int_S F \cdot d\gamma 
:= \int \int_D f\left(\gamma(u,v)\right) 
\cdot \left(\gamma_u \times \gamma_v\right) dA,
$$
where $\gamma_u$ and $\gamma_v$ denote the partial derivatives of $\gamma$
w.r.t. $u$ and $v$ respectively, and $dA$ denotes the standard double integral
over the region $D \subseteq \mathbb{R}^2$.
\begin{itemize}
\item Physical intuition: if $F(x,y,z)$ represents the rate of fluid flow
at a point $(x,y,z)$, then $\int\int_S F(x,y,z) \cdot d\gamma$ gives the
{\it flux} of $F$ over $S$, i.e., the rate at which fluid is flowing out
of $S$ (flux $>0$) or into $S$ (flux $<0$).
\end{itemize}
\end{itemize}

\subsection*{Major Theorems}

The following two theorems establish the fundamental properties of conservative
vector fields as analogues to continuous functions.

\subsubsection*{The Fundamental Theorem of Calculus for Line Integrals}

Given a once continuously differentiable scalar field $f$ ($f\in C^1$)
defined over a curve $C$ parameterized $\gamma(t)$ (for $t\in[a,b]$), 
$\int_C \nabla f \cdot d\gamma = f(\gamma(b)) - f(\gamma(a))$.

Note that as a corollary, it follows that every conservative vector field
$F$ satisfies:
\begin{itemize}
\item Conservation of energy: For a closed curve $C$, 
$\int_C F \cdot d\gamma = 0$.
\item Independence of paths: For any two paths $C_1$ and $C_2$ satisfying 
$\gamma_1(a) = \gamma_2(a)$ and $\gamma_1(b) = \gamma_2(b)$,
$\int_{C_1} F \cdot d\gamma_1 = \int_{C_2} F(\gamma(t)) \cdot d\gamma_2$.
\end{itemize}

\subsubsection*{Conservative Vector Fields}

Let $F(x,y) = P(x,y)\hat{i} + Q(x,y)\hat{j}$ be a a vector field, for which
all first order partials exist and are continuous ($C^1$) in some simply
connected domain $D$. Then, $F$ is conservative if and only if
$\frac{\partial P}{\partial y} = \frac{\partial Q}{\partial x}$ for all 
$(x,y) \in D$.

\bigskip\medskip

The remaining theorems establish a fundamental relationship between the value
of an integral and its boundary, establishing a link between PDEs and
boundary value problems.

\subsubsection*{Green's Theorem}

Let $C$ be a simple closed curve enclosing some region $D$ in $\mathbb{R}^2$.
Then for a vector field $F(x,y) = P(x,y)\hat{i} + Q(x,y)\hat{j}$,
$$
\int_C P(x,y) dx + \int_C Q(x,y) dy = \int\int_D 
\left(\frac{\partial Q}{\partial x} - \frac{\partial P}{\partial y}\right) dA.
$$

\subsubsection*{Stoke's Theorem}

Let $S$ be a positively oriented surface in $\mathbb{R}^3$ (parameterized by 
$\phi(u,v)$) enclosed by a curve $C := \partial S$ (parameterized by $\gamma(t)$).
Let $F(x,y,z) = P(x,y,z)\hat{i} + Q(x,y,z)\hat{j} + R(x,y,z)\hat{k}$ be a vector field.
Then
$$
\int_C F \cdot d\gamma = \int\int_S curl(F) \cdot d\phi.
$$

\subsubsection*{Gauss's (Divergence) Theorem}

Let $E$ be a solid in $\mathbb{R}^3$ (of type 1, 2, or 3), and let $S$ be the surface
bounding $E$ (with positive orientation) parameterized by $\gamma$.
Let $F(x,y,z) = P(x,y,z)\hat{i} + Q(x,y,z)\hat{j} + R(x,y,z)\hat{k}$ be a vector field
with continuous first order partial derivatives.
Then
$$
\int \int_S F\cdot d\gamma = \int\int\int_E div(F) dV.
$$

\end{document}
