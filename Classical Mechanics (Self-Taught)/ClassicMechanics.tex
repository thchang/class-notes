% preamble
\documentclass[12pt]{article}
\usepackage{geometry, amsmath, amssymb, caption, subcaption, enumerate, upgreek}
\geometry{
letterpaper,
total={6.5in,9in},
left=1in,
top=1in}
\parindent = 0pt
\parskip = 6pt

% document
\begin{document}
\section*{Classical Mechanics}
Tyler Chang\\
\today

\subsection*{Basic Definitions}

\begin{itemize}
\item In classical mechanics, a {\it closed system} consists of states and
laws of motion. Given any state, called an {\it initial condition} the laws
of motion should allow one to perfectly predict the future and past.
\begin{itemize}
\item Every state must map to exactly one new state under the laws of motion.
\item That is, the laws must be {\it deterministic} and {\it reversible}.
\item No information can be lost in such a system, i.e., there can be no
singularities.
\end{itemize}
\item A quantity that does not change with time is said to be {\it conserved}.
\item Newton's laws govern most physical systems:
\begin{enumerate}
\item[1.] In the absense of a force, velocity is constant (special case of 2).
\item[2.] $F = m\ddot{x} = \dot{p}$, where $m$ is mass, $x$ is position,
and $p:=m\dot{x}$ is {\it momentum}.
\item[3.] $\sum_{ij} F_{ij} = 0$, where $F_{ij}$ is the force between particles
$i$ and $j$.
It follows, $\frac{d}{dt} p_{total} = F_{total} = 0$, so momentum is conserved.
\end{enumerate}
\item For a force $F$ dependent on position, the {\it potential energy} $V$ is 
given by the potential function $F =: -\nabla V$.
\item The {\it kinetic energy} is given by $T(\dot{x}) := \frac{m}{2}\dot{x}^2$.
\item The total {\it energy} of a system is given by $E := T + V$, and is 
a conserved quantity ($\frac{d}{dt} E = 0$).
\item An {\it inertial reference frame} is a reference frame in which
Newton's laws hold.
\item The {\it phase space} of a system with $n$ coordinates is the 
$2n$ dimensional space obtained by considering both position and momentum.
\end{itemize}

Newton's laws of motion and the corresponding notions of energy and momentum
are only one of many equally legitimate sets of laws and definitions.
In the following sections, we will derive several frameworks in which
all classical mechanics can be cast.

\subsection*{The Principle of Stationary Action and Lagrangian}

The {\it action} of a system is given by the functional
$$
S[q(t)] := \int_{t_1}^{t_f} L[q(t),\dot{q}(t),t] dt,
$$
where $L$ is called the {\it Lagrangian} and the integral is taken over
some path through phase space.
{\it The principle of stationary action} states that the laws of motion are
given by the path through phase space for which the action is {\it stationary} 
(either a minimum, maximum, or saddle point) for fixed start and end points
$q(t_1)$ and $q(t_f)$.

Given an arbitrary Lagrangian, one can solve for the resulting path using 
{\it calculus of variations}.
The solution is given by the {\it Euler-Lagrange equations}:
$$
\frac{\partial L}{\partial x}=\frac{d}{dt}\frac{\partial L}{\partial\dot{x}}.
$$
Of course, for an arbitrary Lagrangian, the notion of momentum may not
be well-defined, so we define the momentum by
$$
p := \frac{\partial L}{\partial \dot{x}} \qquad\text{ and it follows that } 
\qquad\dot{p} = \frac{\partial L}{\partial x}.
$$

A continuous transformation is given by any transformation of the form 
$$\delta x_i = f_i(x) \varepsilon$$
where $\delta x_i$ is the change in the $i$th component of position and $f_i$ is
any continuous function of position. 
Such a tranformation is called a {\it symmetry} if it doesn't affect the 
Lagrangian ($\delta L = 0$).
Given a symmetry, there exists a quantity $Q = \sum_i p_i f_i$ that
is conserved ($\frac{d}{dt}Q = 0$).
For example, if the Lagrangian is rotationally invariant (rotation
is a symmetry), then angular momentum is a conserved quantity.

To recover Newton's equations, one can use the following Lagrangian 
$$
L(x, \dot{x}) = T(\dot{x}) - V(x)
$$
where $T(\dot{x})$ and $V(x)$ are the kinetic and potential energies as defined
in Newton's formula.
More generally, for an arbitrary Lagrangian, if $L$ can be expressed as the
difference between a quadratic function of $\dot{x}$ and a function of $x$,
we can define the kinetic energy to be the function of $\dot{x}$ and the
potential energy to be the function of $x$.

\subsection*{The Hamiltonian}

The Hamiltonian is given by
$$
H(x, \dot{x}, t) := \langle p, \dot{x} \rangle - L(x, \dot{x}, t)
$$
where $\langle \cdot , \cdot \rangle$ denotes the dot product.
In systems where there exists a notion of energy, the energy is defined
to be the Hamiltonian.
If the Lagrangian has no explicit dependence on time ($L(x, \dot{x})$) then
the Hamiltonian/energy is conserved.

Hamilton's equations of motion are then given by
$$
\dot{p} = \frac{\partial H}{\partial x} \qquad\text{ and }\qquad
\dot{x} = \frac{\partial H}{\partial p}.
$$

Recall that a flow is said to be {\it incompressible} if the volume of inflow
is equal to the volume of outflow for all regions, or equivalently, if 
the flow maps regions to regions of equal volume.
This is the analogue to determinism and reversibility in a continuous state
space.

\underline{Liouville's Theorem}: The flow in phase space (for any Hamilton
equations) is incompressible, i.e., $div(V) = 0$, where $V$ is the vector field
describing the laws of motion in phase space for some Hamiltonian.

One final important piece of notation is the {\it Poisson bracket}.
The brackets are defined for two functions $F$ and $G$ in phase space:
$$
\{F,G\} := \sum_i \left(
\frac{\partial F}{\partial x_i}\frac{\partial G}{\partial p_i} -
\frac{\partial F}{\partial p_i}\frac{\partial G}{\partial x_i} \right).
$$
If $F(p,x)$ is a force in some system governed by a Hamiltonian $H$, then 
$\dot{F} = \{F, H\}$.
Also, if $G$ is a conserved quantity, then $\{G, H\} = 0$.

\subsection*{Electrostatics}

Magnetic fields (in the absense of monopoles) are incompressible 
vector fields that describe a force on charged particles.
The {\it Lorentz equation} describes the forces on a charged particle in
the presence of a stationary electromagnetic field:
$$
F = \frac{e}{c}\left(-\nabla V + (\dot{x} \times B)\right)
$$
where $E = -\nabla V$ is the electric field, $B$ is the magnetic field,
$e$ is the charge of the particle, and $c$ is the speed of light.

Note that since $B$ is incompressible, $B$ can be expressed as the curl
of some other field $B = \nabla \times A$, where $A$ is called the
{\it vector potential}.
Of course, since $\nabla \times \nabla S = 0$ for any scalar field $S$
(called the Gauge),
$A$ is not unique and can be {\it Gauge transformed} without affecting
the magnetic field $B$.
However, the choice of Gauge does not affect the equations of motion so
this is not an issue.

The Lagrangian that yields Lorentz's equations of motion is
$$
L(x, \dot{x}) = \frac{m}{2}\dot{x}^2 + \frac{e}{c}(A \cdot \dot{x}).
$$

\end{document}
