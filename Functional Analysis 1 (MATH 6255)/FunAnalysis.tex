% preamble
\documentclass[12pt]{article}
\usepackage{geometry, amsmath, amssymb, caption, subcaption, enumerate}
\geometry{
letterpaper,
total={6.5in,9in},
left=1in,
top=1in}
\parindent = 0pt
\parskip = 6pt

% document
\begin{document}
\section*{Functional Analysis}
Tyler Chang\\
\today

\subsection*{Banach and Hilbert Spaces}

A {\it norm space} $(X,\|.\|)$ is a vector space $X$ together with a norm
$\|.\|$. Such a space is called a {\it Banach space} if $X$ is 
{\it complete} under the metric topology induced by $d(x,y) = \|x - y\|$.
Note that every norm space has a {\it completion} that is Banach. Also
note that in a finite dimensional space, all norms induce equivalent
topologies since for every pair of norms $\|.\|_1$ and $\|.\|_2$ and
all points $x\in X$, there exists a uniform constant $1 < c< \infty$
such that $\frac{1}{c}\|x\|_2 \leq \|x\|_1 \leq c\|x\|_2$.

A {\it linear functional} $\ell$ on $(X,\|.\|)$ is a linear
function $\ell : X \rightarrow \mathbb{R}$ or $\mathbb{C}$.
$\ell$ is said to be {\it bounded} if $|\ell(x)| \leq c\|x\|$
for all $x\in X$ and $c < \infty$.
Equivalently, $\ell$ is bounded if and only if $\ell$ is continuous.
Every norm space $(X, \|.\|_X)$ has a {\it dual} $(X', \|.\|_{X'})$
given by the vector space $X'$ of all bounded linear functionals $\ell$
on $X$, together with the norm
$$
\|\ell\|_{X'} = {\sup \atop {x\neq 0 \atop x\in X}} \frac{|\ell(x)|}{\|x\|_X}
= {\sup \atop \|x\|_X = 1} |\ell(x)|.
$$
In the special case of $L^p$ (or $\ell^p$), we have for $1\leq p < \infty$,
$(L^p(X))' = L^q(X)$ where $\frac{1}{p} + \frac{1}{q} = 1$ (Riesz
Representation). It immediately follows that $L^p$ is {\it reflexive} 
(meaning $(L^p)'' = L^p$) for $1 < p < \infty$, and $(L^2)' = L^2$.
Furthermore, we have the H{\"o}lder Inequality: If 
$\frac{1}{r} = \frac{1}{q} + \frac{1}{p}$ and $a\in\ell^p$, $b\in\ell^q$,
then $\|ab\|_r \leq \|a\|_p \|b\|_q$.

An {\it inner product} or {\it scalar product} space 
$(X, \langle \cdot\text{, }\cdot \rangle)$ is a vector space $X$
together with a scalar product 
$\langle\cdot\text{, }\cdot\rangle : X \times X \rightarrow F$
(where $F = \mathbb{R}$ or $\mathbb{C}$) that satisfies
\begin{itemize}
	\item Bilinearity: If $F = \mathbb{R}$, then
		$\langle\cdot\text{, }\cdot\rangle$ is linear in the first
		and second arguments. If $F = \mathbb{C}$, then
		$\langle\cdot\text{, }\cdot\rangle$ is linear in the first
		argument and {\it sesquilinear} in the second
		($\langle x,\alpha y\rangle={\bar\alpha}\langle x,y\rangle$).
	\item Symmetry/Skew Symmetry: If $F = \mathbb{R}$, then
		$\langle x, y\rangle = \langle y, x\rangle$.
		If $F = \mathbb{C}$, then
		$\langle x, y\rangle = \overline{\langle y, x\rangle}$.
	\item Positivity: $\langle x,x \rangle > 0$ if $x\neq 0$
		(and $\langle 0, 0 \rangle = 0$ by linearity).
\end{itemize}
Every scalar product induces a norm $\|x\|^2 = \langle x, x\rangle$.
If $X$ is complete under the topology induced by this norm (i.e., 
$(X, \|.\|)$ is Banach), then $(X, \langle\cdot\text{, }\cdot\rangle)$
is called a {\it Hilbert space}. Note that since $(L^2)' = L^2$,
$L^2$ is a Hilbert space under $\langle f, g \rangle = \int f{\bar g}$.
In fact, every Hilbert space is isomorphic to its dual.

Every inner product $\langle\cdot\text{, }\cdot\rangle$ that induces a
norm $\|.\|$ satisfies the following identities:
\begin{itemize}
	\item Cauchy-Schwarz: $|\langle x,y\rangle| \leq \|x\|\|y\|$.
	\item Parallelogram Law: $\|x+y\|^2+\|x-y\|^2=2(\|x\|^2+\|y\|^2)$.
	\item Polarization ID: 
		$\langle x,y\rangle=\frac{1}{4}\sum_{k=0}^3 i^k\|x+i^k y\|^2$.
\end{itemize}
Furthermore, given a norm space $(X, \|.\|)$, there exists an inner product
$\langle\cdot\text{, }\cdot\rangle$ such that $\langle x,x\rangle = \|x\|^2$
if and only if $\|.\|$ satisfies the parallelogram law. Furthermore, that
inner product is given by the polarization identity.

Given a set of vectors $S$ in a Hilbert space $H$, the span of $S$ is
given by the set of all finite linear combinations of elements in $S$.
I.e., span~$S$ is the smallest subspace of $H$ containing $S$.
Note that in the infinite-dimensional case, span~$S$ need not be
closed (as it is in the finite dimensional case).
A collection of vectors $S$ in $H$ is said to be {\it orthonormal}
if their {\it projections} $\langle y_\alpha, y_\beta\rangle = 0$
($y_\alpha \perp y_\beta$)
for all $\alpha\neq\beta$ and $\langle y_\alpha, y_\alpha\rangle =1$.
An orthonormal set is called an orthonormal {\it basis} if the 
closure of its span satisfies $\overline{\text{span}}$~$S=H$.
It follows that for any orthonormal set $S$:
\begin{itemize}
	\item (Parseval's ID) If $S$ is basis for $H$, then 
		$\|x\|^2=\sum_{e_\alpha\in S}|\langle x,e_\alpha\rangle|^2$
		for all $x\in H$.
	\item (Bessel's Ineq) In general, 
		$\|x\|^2\geq\sum_{e_\alpha\in S}|\langle x,e_\alpha\rangle|^2$
		for all $x\in H$.
\end{itemize}
Every Hilbert space has a basis, which can be orthogonalized
using the Gram-Schmidt process. So, by identifying basis vectors,
we see that two Hilbert spaces are isomorphic if and only if their
bases have equal cardinalities. It follows that every Hilbert space
of countably infinite dimension is isomorphic to $L^2$.

\subsection*{Important Theorems}

{\bf Zorn's Lemma}

Let $S$ be a partially ordered set (poset) under a relation ($\leq$)
such that every totally ordered subset has an upper bound. Then $S$
has a maximal element under the relation ``$\leq$.''

\underline{Notes on Zorn's Lemma}:\\
In the infinite-dimensional case, Zorn's lemma is often used in place
of induction in a process called ``Zornification.''
\bigskip

{\bf Hahn-Banach Theorem}

Let $X$ be a vector space over $F = \mathbb{R}$ or $\mathbb{C}$ and
let $p$ be a positive real-valued function on $X$ that
satisfies
\begin{itemize}
	\item Positive homogeneity: for $a\in\mathbb{R}_+$ (where in the
		complex case, $\mathbb{R}_+$ denotes the positive half-plane)
		and $x\in X$, $p(ax) = |a| p(x)$.
	\item Sub-additivity: for $x,y\in X$, $p(x+y) \leq p(x) + p(y)$.
\end{itemize} 
Let $Y$ be a subspace of $X$ and let $\ell : Y \rightarrow F$ be a linear
functional on $Y$ such that $|\ell(y)| \leq p(y)$ for all $y\in Y$
($p$ dominates $\ell$).
Then $\ell$ can be extended to a linear functional on $X$ that is still
dominated by $p$.
\bigskip

{\bf Clarkson's Theorem}

Let $X$ be a norm space that is uniformly convex.
Also let $z\in X$ and let $K$ be a closed convex subset of $X$.
Then there exists a unique point $y_0\in K$ that solves:
$$
\|x - y_0\| = \inf_{y\in K} \|z - y\| = dist(z,k).
$$\bigskip

{\bf Riesz's Lemma}

Let $Y$ be a closed proper subspace of a norm space $X$.
Then there exists a unit vector $z\in X$ such that for all $y\in Y$,
$\|z - y\| \geq \frac{1}{2}$.

\underline{Notes on Riesz's Lemma}:\\
It follows from Riesz's Lemma that in an infinite-dimensional space,
the closed unit ball is not compact (as it is in the finite-dimensional
case). This inspires the {\it weak topology}, the weakest topology in 
which all $\ell\in X'$ are continuous. I.e., $x_n \rightarrow x$ weakly
if $|\ell(x_n) - \ell(x)| \rightarrow 0$ for all $\ell \in X'$.
Under the weak topology, the closed unit ball is indeed compact.
Furthermore, if $x_n \rightarrow x$ in the weak topology, then 
$\{x_n\}$ either does not converge in the norm topology, or
$\|x_n - x\| \rightarrow 0$.
\bigskip

{\bf Baire Category Theorem}

Let $X$ be a complete metric space and let $X = \cup_{n=1}^\infty S_n$
(where $\{S_n\}$ is countable). Then at least one $S_n$ is {\bf not}
nowhere dense.
\bigskip

{\bf Principle of Uniform Boundedness}

Let $X$ be a complete metric space and let $\mathcal{F}$ be a collection
of real-valued continuous functionals on $X$.
If $\mathcal{F}$ is {\it pointwise bounded} ($|f(x)|\leq M(x)$ for all 
$x\in X$ and $f\in\mathcal{F}$), then there exists an open set $U\subset X$
in which $\mathcal{F}$ is {\it uniformly bounded}
($|f(x)|\leq M<\infty$ for all $x\in U$ and $f\in\mathcal{F}$).
In the special case where $\mathcal{F}$ are sub-additive and absolutely
homogeneous (most commonly, when $\mathcal{F}$ are linear) and $X$ is
Banach with norm $\|.\|$, $\mathcal{F}$ is uniformly bounded on the entire
space $X$:
$|f(x)| \leq c\|x\|$ for $c <\infty$.
\bigskip

{\bf Open Mapping Theorem}

Every bounded linear map $M : Y \rightarrow Y$ is {\it open}
(If $U$ is open, so is $M(U)$).
\bigskip

{\bf Closed Graph Theorem}

Let $X$ and $Y$ be Banach and let $M : X \rightarrow Y$ be a linear
operator. We call $\Gamma(M) = \{(x, Mx) : x\in X\}$ the {\it graph}
of $M$. We say $M$ is {\it closed} if its graph is closed in
$X\times Y$ under the norm $\|(x,y)\| = \|x\|_X + \|y\|_Y$.
The closed graph theorem says that if $M$ is closed, then $M$ is
bounded.

\subsection*{Bounded Linear Operators}

Let $T$ be a linear operator $T : X \rightarrow Y$, where $X$, $Y$
are Banach.
Then $T$ is bounded if 
\begin{equation}
	\|T\| = {\sup\atop {x\in X \atop x\neq 0}}
	\frac{\|Tx\|_Y}{\|x\|_X}
	= {\sup \atop \|x\|_X = 1} \|Tx\|_Y 
\end{equation}
is bounded by some constant $M$. Furthermore, we consider the Banach space
$\mathcal{L}(X,Y)$ of all bounded linear functionals from $X$ to $Y$ under
the norm defined in (1). Other useful topologies on $\mathcal{L}(X,Y)$
include the {\it strong operator topology} (the weakest topology in which 
$T : X \rightarrow Y$ is continuous for all $T\in\mathcal{L}(X,Y)$) and
the {\it weak operator topology} (the weakest topology in which 
$T : X \times Y' \rightarrow F$ is continuous for all 
$T\in\mathcal{L}(X,Y)$).

If $V \subset X$ is a subspace of $X$ then
\begin{itemize}
	\item The {\it annihilator} $V^\perp$ of $V$ is given by the set
		of all linear functionals $\ell \in X'$ that vanish on
		$V$. I.e., for all $\ell \in V^\perp$ and $y \in V$,
		$\ell(y) = 0$.
	\item $X / V$ denotes the {\it quotient space}, i.e., the space of
		equivalence classes $[x]$ such that $[x] = [z]$ for
		$x,z\in X$ and $x-z \in V$.
\end{itemize}

If $X$, $Y$ are Hilbert spaces, then we can define the {\it transpose} of
a linear operator $T$ as the operator $T' \in \mathcal{L}(Y',X')$ such that
$\langle x, T'\ell\rangle = \langle Tx, \ell \rangle$ for all $x\in X$ and
$\ell\in Y'$. Similarly, the {\it Hermitian adjoint} of $T$ is the operator
$T^* \in \mathcal{L}(Y,X)$ such that 
$\langle T^* x, y\rangle = \langle x, Ty \rangle$ for all $x\in X$ and
$y\in Y$. Such an operator $T\in\mathcal{L}(X,X)$ is said to be {\it symmetric}
or {\it self-adjoint} if $T^* = T$. 

In the special case where $X = Y$, we denote $\mathcal{L}(X,X)$ by
$\mathcal{L}(X)$. A few notable linear operators in 
$\mathcal{L}(L^2(\mathbb{R}))$ and $\mathcal{L}(L^2(\mathbb{R}_+))$ are
the Fourier and Laplace transforms respectively. If an operator
$T \in \mathcal{L}(X)$ is bijective, then it follows by the open
mapping theorem that its inverse $T^{-1}$ exists. The space
$\mathcal{GL}(X)$ of all invertible operators on $X$ is itself a
subspace of $\mathcal{L}(X)$, and furthermore, $\mathcal{GL}(X)$ is a
two-sided ideal under multiplication. Let $K \in \mathcal{L}(X)$
such that $K = I + E$, where $\|E\| < 1$ and $I$ denotes the identity
on $X$. Then $K \in \mathcal{GL}(X)$ and its inverse is given by the
{\it Neumann series} $K^{-1} = \sum_{n=0}^\infty (-1)^n E^n$, which
converges geometrically.

An operator $C\in\mathcal{L}(X,Y)$ is said to be {\it compact} if $C(B_1(0))$
(where $B_1(0)$ denotes the unit ball in $X$) is precompact in $Y$. The
space $\mathcal{C}(X,Y)$ of compact operators between $X$ and $Y$ is itself
a closed proper subspace of $\mathcal{L}(X,Y)$. Let $C \in \mathcal{C}(X)$
and let $T = I - C$, where $I$ denotes the identity on $X$. Then the $T$
is invertible and its Neumann series converges. Furthermore, if $N_T$
denotes the nullspace of $T$ and $R_T$ denotes the range of $T$, then
\begin{itemize}
	\item The sequence $N_{T^n}$ $k=1,\ldots,\infty$ saturates at
		some finite value
		$N_{T^N} = \lim_{n\rightarrow\infty} N_{T^n}$.
	\item The dimension of $N_T$ is given by 
		$\text{dim }N_T = \text{dim }(X /R_T)$.
\end{itemize}
These statements hold trivially in the finite-dimensional case (the
second statement is equivalent to the statement that dim~$X = R_T + N_T$).
However, in the infinite-dimensional case they generally fail. For a general
operator $K \in \mathcal{L}(X)$, the discrepancy dim~$N_T - $~dim$(X/R_T)$ is
called the {\it Fredholm index} of $K$.

\subsection*{Spectral Theory}

For a general operator $T \in \mathcal{L}(X)$, the {\it resolvent set} is
given by $\rho(T)=\{\lambda\in\mathcal{C}:(\lambda I-T)\in\mathcal{GL}(X)\}$.
The {\it spectrum} of $T$ is then given by 
$\sigma(T) = \mathbb{C}\setminus\rho(T)$. For every
operator, the spectrum satisfies $\sigma(T)\subseteq \overline{B_{\|T\|}(0)}$
where $\overline{B_{\|T\|}(0)}$ denotes the closed ball of radius $\|T\|$
centered at the origin. Furthermore, $\sigma(T)$ is always closed and
nonempty. Unfortunately, for a general operator $T\in\mathcal{L}(X)$,
$\sigma(T)$ can be uncountable and may contain elements that are {\bf not}
eigenvalues for $T$. This makes the spectrum difficult to work with for
a general bounded linear operator.

However, for a compact operator $C \in \mathcal{C}(X)$, every
$\lambda_j \neq 0$ in $\sigma(C)$ {\bf is} an eigenvalue of $C$, the
spectrum of $C$ is at most countably infinite, and the only possible
accumulation point for $\lambda_j \in \sigma(C)$ is $0$. Furthermore,
if $C$ is self-adjoint and compact, we can conclude that $C$ is
diagonally dominated in the sense that $\|T\| = \max(\lambda_+, -\lambda_-)$
where $\lambda_+={\sup\atop\|u\|=1}\langle Tu,u\rangle$
and $\lambda_-={\inf\atop\|u\|=1}\langle Tu,u\rangle$.
Furthermore, the sup and inf are both attained, and
$\lambda_+ = \max(\sigma(T))$, $\lambda_- = \min(\sigma(T))$.
It follows that for $C\in\mathcal{C}(X)$ where $X$ is a separable Hilbert
space and $C$ is self-adjoint, $C$ has an {\it eigenbasis}, i.e., there
exists an orthonormal basis $\{x_n\}_{n=1}^\infty$ for $X$ such that
$Cx_n = \lambda_n x_n$. For example, the Laplace transform is a compact
self-adjoint operator in $\mathcal{C}(L^2(\mathbb{R}_+))$, with
$\lambda_+ = \sqrt{\pi}$ where $(\sqrt{\pi},t^{-1/2})$ is an Eigenpair.

More generally, for $C\in\mathcal{C}(X)$ where $X$ is a Hilbert space and
$C=C^*$, we have the {\it max-min} and {\it min-max principles}:\\
Let $S_n$ denote any $n$-dimensional subspaces of $X$. Then if
$$
\alpha_n=\max_{S_n}\min_{u\in S_n\atop\|u\|=1}\langle Tu,u\rangle,
\qquad\text{\bf or }\qquad
\alpha_n=\min_{u\in S_n\atop\|u\|=1}\max_{S_n}\langle Tu,u\rangle,
$$
then $\alpha_1 \geq \alpha_2 \geq \ldots \geq 0$, and all non-zero
$\alpha_j$ are eigenvalues of $T$.

\end{document}
