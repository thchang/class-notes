% preamble
\documentclass[12pt]{article}
\usepackage{geometry, amsmath, amssymb, caption, subcaption, enumerate}
\geometry{
letterpaper,
total={6.5in,9in},
left=1in,
top=1in}
\parindent = 0pt
\parskip = 6pt

% document
\begin{document}
\section*{Real Analysis 2}
Tyler Chang\\
\today

\subsection*{Differentiation Theory}

To define the derivative of a function $F$, consider {\it Dini's derivatives}
which exist for all $x$:
$$ \begin{array}{ccc}
({\overline D}^+ F)(x) := \limsup_{h\rightarrow 0^+} \frac{F(x+h) - F(x)}{h} & &
({\underline D}^+ F)(x) := \liminf_{h\rightarrow 0^+} \frac{F(x+h) - F(x)}{h} \\
& \\
({\overline D}^- F)(x) := \limsup_{h\rightarrow 0^+} \frac{F(x) - F(x-h)}{h} & &
({\underline D}^- F)(x) := \liminf_{h\rightarrow 0^+} \frac{F(x) - F(x-h)}{h}.
\end{array} $$
If all four Dini's derivatives are equal, then we say that the {\it derivative}
$F'(x)$ exists and
$$F'(x) := \lim_{y\rightarrow x} \frac{F(y) - F(x)}{|y-x|}.$$
If $F : [a,b] \rightarrow \mathbb{R}$ is differentiable for all $x\in[a,b]$, 
then we say that $F$ is {\it differentiable}.
Similarly, if $F$ is differentiable except on a null set, we say $F$ is 
{\it almost everywhere differentiable}.

First, consider which classes of functions are a.e. differentiable?
Measurability, boundedness, and even continuity are not enough to ensure 
differentiability.
As a counterexample, the Weierstrass function
is bounded and continuous, but nowhere differentiable:
$$
F(x) = \sum_{n=1}^\infty a^{-n} \cos(b^n\pi x) 
\qquad\text{for $a > 1$ and $\frac{b}{a} > 1 + \frac{3}{2}\pi$.}
$$
It is the oscillations of the cosine function that break differentiability.
In fact, all measurable monotone functions are a.e. differentiable 
({\it Monotone differentiation theorem}).
However, a stronger statement can be made.

Let the {\it total variation} of $F$ be given by:
$$
\|F\|_{TV(\mathbb{R})} 
:= \sup_{x_0<x_1<\ldots<x_n} \sum_{i=1}^n |F(x_i) - F(x_{i-1})|
$$
where the supremum is taken over all finite increasing sequences.
A function $F$ is said to have {\it bounded variation} ($F \in BV(\mathbb{R})$)
if $\|F\|_{BV} := \|F\|_{TV} = M < \infty$.
Every function of bounded variation can be decomposed into the difference
between two bounded monotone nondecreasing functions: $F = F^+ - F^-$.
It follows that every $f\in BV(R)$ is a.e. differentiable.
Note: every Lipschitz continuous function $f$ satisfies $f\in BV(\mathbb{R})$.

We would like to extend the two {\it fundamental theorems of calculus} to the
a.e. differentiable case using Lebesgue integrals instead of Riemann integrals.
However, the theorems are harder to prove in the Lebesgue case, given that for 
$F$ a.e. differentiable, both {\it Rolle's theorem} and the 
{\it Mean Value Theorem} in general fail.

The analogue of the first fundamental theorem is:

\hangindent 12pt
\underline{The Lebesgue Differentiation Theorem}:
Let $f : \mathbb{R}\rightarrow\mathbb{C}$ be $L^1$ and let 
$F : \mathbb{R}\rightarrow\mathbb{C}$ be the Lebesgue integral:
$F(x) = \int_{-\infty}^x f(t) dt$.
Then $F$ is continuous and a.e. differentiable, and $F'(x) = f(x)$ for
a.e. $x\in\mathbb{R}$.

The 2nd FTC is much harder to show.
Indeed, it fails in general for both monotone and continuous functions.
Instead, we must introduce a new notion that is stronger than both 
{\it continuity} and {\it uniform continuity}:

A function $F : \mathbb{R} \rightarrow \mathbb{R}$ is 
{\it absolutely continuous} ($F\in AC(\mathbb{R})$) if 
for all $\varepsilon >0$, there exists $\delta > 0$ such that
for all collections of disjoint intervals $\{(a_j, b_j)\}$ that satisfy
$\sum_{j=1}^n |b_j - a_j| \leq \delta$,
$\sum_{j=1}^n |F(b_j) - F(a_j)| \leq \varepsilon$.

\hangindent 12pt
\underline{2nd FTC for AC Functions}:
Let $F \in AC([a,b])$ for all compact $[a,b] \subset \mathbb{R}$.
Then $F'$ exists and $\int_a^b F' = F(b) - F(a)$.
Conversely, for $f \in L^1([a,b])$, there exists an $F \in AC([a,b])$
such that $F' = f$ a.e. and $F(x) = \int_a^x f$.

\subsection*{$L^p$ Spaces}

Extending the notion of the $L^1$ class, define the class of $L^p$ functions
as follows for any $p$:
$$
f \in L^p(X) \iff \int_X |f(x)|^p d\mu < \infty.
$$
For $p > 1$, the exponent $p$ increases the rate at which $f$ decays, making
this a broader class than $L^1$.
Taking the limit as $p \rightarrow \infty$, we say that $f\in L^\infty$
if $|f|$ is bounded by some $M < \infty$ almost everywhere.
For $p \geq 1$, $L^p$ is a {\it norm space} under the norm 
$$
\|f\|_{L^p} = \left(\int |f|^p \right)^{1/p}.
$$
If $p < 1$, then $L^p$ is only a {\it quasi-norm space}.

Under the metric induced by $\|.\|_{L^p}$, both the 
addition $+ : L^p \times L^p \rightarrow L^p$ and multiplication 
$\cdot : \mathbb{C} \times L^p \rightarrow L^p$ operations are jointly 
continuous (under the product topology).
Therefore, we say that $L^p$ is a {\it toplogical vector space}.
Furthermore, $L^p$ is actually an infinite-dimensional {\it Banach space},
i.e., a {\it complete metric space}, and satisfies the {\it Hausdorff property}.
Put in practical terms, this means that for every {\it Cauchy sequence} in an 
$L^p$ space, the limit exists and is unique.

The {\it Riesz Representation Theorem} tells us about the {\it dual space}
$\left( L^p \right)^*$.
For $\frac{1}{p} + \frac{1}{q} = 1$, $(L^p)^* = L^q$.
That is, for any {\it linear functional} $\ell \in (L^p)^*$, there exists a
unique $g \in L^q$ such that $\ell(f) = \int f \overline{g}$.
For $1 < p < \infty$, $L^p$ space is {\it reflexive}, meaning 
$(L^p)^{**} = L^p$.
Furthermore, in the special case where $p = 2$, $(L^2)^* = L^2$.
Therefore, $L^2$ space is a {\it Hilbert space} (complete inner product space)
under the inner product
$$
\langle f, g \rangle := \int f \overline{g}
\qquad\text{for all $f,g \in L^2$.}
$$

\subsection*{Basic Functional Analysis}

One of the fundamental theorems of functional analysis is:

\hangindent 12pt
\underline{The Baire Category Theorem} (BCT): Let $(X,d)$ be a complete metric
space, and let $\{E_n\}_{n=1}^\infty$ be a countable sequence of subsets of
$X$.
If $\cup_{n=1}^\infty E_n \supset B$ for any ball $B$ in $(X,d)$, then there
exists a nonempty sub-ball $B' \subset B$ in which at least one $E_n$ is dense.

Using BCT, we categorize topological sets.
Note, that the analogies are informal:

\begin{tabular}{|l|l|l|}
\hline
Name of set & Definition & Analogy in a measure space\\
\hline
Meager (1st category) & An at most countable union of & null sets\\
 & nowhere dense sets & \\
\hline
2nd category & Any set that is not 1st category & sets of positive measure \\
\hline
Co-meager (residual) & $E$ is co-meager if $E^C$ is meager & full-measure sets\\
\hline
Baire (almost open) & $E$ is Baire if $E \bigtriangleup U$ is meager & 
measurable sets\\ 
& for some open set $U$ & \\
\hline
\end{tabular}

BCT has several important corollaries involving {\it linear operators}
(linear maps $T : X \rightarrow Y$ between norm spaces).
If $T$ satisfies $\|T\| := \sup_{x\in X}\frac{\|Tx\|_Y}{\|x\|_X} = M < \infty$,
then $T$ is called a {\it bounded linear operator}.

\hangindent 12pt
\underline{The Uniform Boundedness Principle} (UBP):
Let $X$ be a Banach space and $Y$ be a general norm space, and let
$\{T_\alpha\}_{\alpha\in A} : X \rightarrow Y$ be a family of bounded linear 
operators.
Then the following are equivalent:
\begin{itemize}
\item (Pointwise boundedness) For all $x\in X$, the collection in $Y$ given by
$\{T_\alpha x : \alpha \in A\}$ is bouned in $Y$-norm.
\item (Uniform bounedness) The operator norms $\{\|T_\alpha\| : \alpha\in A\}$
are uniformly bounded (i.e., they are all bounded by some finite constant).
\end{itemize}

For a general function $f : X \rightarrow Y$ where both $X$ and $Y$ are
topological spaces, $f$ is an {\it open map} if for all open $U \in X$, 
$f(U)$ is open in $Y$.
That is, $f$ maps open sets to open sets.
Contrast this with a {\it continuous map}, where $f^{-1}(U)$ is open in $X$
for all $U$ open in $Y$.

\hangindent 12pt
\underline{The Open Mapping Theorem} (OMT):
Let $T : X \rightarrow Y$ be a bounded linear operator between Banach
spaces $X$ and $Y$.
Then the following are equivalent:
\begin{itemize}
\item $T$ is onto (surjective).
\item $T$ is open.
\item (Qualitative solvability) For all $y\in Y$, there exists $x\in X$ such
that $Tx = y$.
\item (Quantitative solvability) For all $y\in Y$, there exists $x\in X$ such
that $Tx = y$ and $\|x\|_X \leq M\|y\|_Y$ for some constant $M$.
\item (Quantitative solvability on a dense subclass) There exists a constant
$M > 0$ such that for any dense subset $E$ of $Y$, there exists $x\in X$
such that $Tx = y$ for all $y \in E$ and $\|x\|_X \leq M\|y\|_Y$.
\end{itemize}

\hangindent 12pt
\underline{The Closed Graph Theorem} (CGT):
Let $T:X\rightarrow Y$ be a linear operator between Banach spaces.
Then the following are equivalent:
\begin{itemize}
\item $T$ is continuous.
\item $T$ is closed: I.e., the graph $\Gamma := \{(x,Tx) : x\in X\}$
is closed under the product topology $X \times Y$.
\item $T$ is weakly continuous: I.e., there exists some topology $\mathcal{F}$
on $Y$, weaker than the norm topology but still Hausdorff, such that
$T : X \rightarrow (Y,\mathcal{F})$ is continuous.
\end{itemize}

\subsection*{Infinite Dimensional Spaces and Weak Topologies}

In an infinite dimesional space (such as $L^p(\mathbb{R})$), it is
difficult to formulate compactness arguments due to the failure of
the {\it Heine-Borel Theorem}, which tells us that
in a finite dimensional metric space, 
the following definitions of compactness are equivalent.
$X$ is compact if:
\begin{itemize}
\item (Compactness) Every open cover of $X$ has a finite subcover.
\item (Sequential Compactness) Every sequence $\{x_n\}$ in $X$ has a convergent 
subsequence.
\item (Closed and Boundedness) $X$ is closed and bounded.
\end{itemize}

To see what can go wrong in an infinite dimensional space, consider the 
closed unit ball $B$.
Intuitively, since $B$ is closed and bounded, it should be compact.
However, in an infinite-dimesional space, one can construct an infinite
sequencei $\{e_j\}$, each of whose elements is the unit basis vector in a new 
dimension.
Then no subsequence of $\{e_j\}$ converges.

To recover a friendlier topology in infinite-dimensional space, consider
the following weaker topology:
Let $(X,\|.\|)$ be a normed space.
For each $\ell \in X^*$ (bounded), define the seminorm 
$\rho_\ell(x) := |\ell(x)|$.
Then the {\it weak topology} is the topology generated by all open
strips/balls of the form: $B_r^\ell(x):=\{y\in X: \rho_\ell(x-y) < r\}$.
Similarly, the weak$^*$ topology on $X^*$ is generated by all seminorms
of the form $\rho_x(\ell) := |\ell(x)|$.
Under the weak topology on a reflexive separable Banach space $X$,
the closed unit ball is indeed sequentially compact.

We have already been briefly introduced to the {\it norm operator topology} 
(NOT).
Let $X,Y$ be Banach spaces and let $B(X,Y)$ be the collection of all 
bounded linear operators $T : X \rightarrow Y$.
Then $B(X,Y)$ is a norm space under:
$$
\|T\| := \sup_{x\in X\text{, }x\neq 0}\frac{\|Tx\|_Y}{\|x\|_X}
\qquad\text{for all $T\in B(X,Y)$}.
$$
Similarly as above, we also introduce the {\it strong operator topology}
(SOT) and {\it weak operator topology} (WOT).
SOT is the topology induced by seminorms $\rho_x(T) := \|Tx\|$ for all
$x\in X$,
and WOT is the topology induced by seminorms 
$\rho_{x,\ell}(T) := |\ell(Tx)|$ for all $x\in X$ and $\ell \in Y^*$.

SOT and WOT are generally used in arguments of convergence:
\begin{align*}
T_n \rightarrow T \text{ in SOT iff: }& \|T_n x - Tx\| \rightarrow 0
\text{ for all $x \in X$.}\\
T_n \rightarrow T \text{ in WOT iff: }& |\ell(T_n x) - \ell(Tx)| \rightarrow 0
\text{ for all $x \in X$, $\ell\in Y^*$.}
\end{align*}
In order of strength, convergence in NOT is stronger than in SOT is stronger 
than in WOT.

\end{document}
